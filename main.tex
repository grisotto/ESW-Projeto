\documentclass[12pt,a4paper]{article}
\usepackage[utf8]{inputenc} % sempre salve seus arquivos como UTF8
\usepackage[T1]{fontenc}
\usepackage[brazil]{babel}
%Paulo - adicona esta linha
\usepackage[backend=biber, doi=true, maxbibnames=9, maxcitenames=2]{biblatex}


\usepackage[left=2.5cm,right=2cm,top=2cm,bottom=2.5cm]{geometry}
\usepackage{amsmath}
\usepackage{amsthm}
\usepackage{graphicx}
\usepackage{algorithm}
\usepackage{color}
\usepackage[noend]{algpseudocode}
\usepackage{mathtools}
\usepackage{fancybox}
\usepackage{tikz}

% load times font
\usepackage{mathptmx}
\usepackage[scaled=.90]{helvet}
\usepackage{courier}
%Qual linha?
\bibliography{general.bib}

% teoremas, lemas, etc...
\newtheorem{questao}{Questão}
\newtheorem{invariant}{Invariante}
\newtheorem{theorem}{Teorema}
\newtheorem{lemma}[theorem]{Lema}
\renewcommand*{\proofname}{Demonstração}

% comandos
\newcommand{\mdc}[1]{\mathrm{mdc}(#1)}
\DeclarePairedDelimiter\ceil{\lceil}{\rceil}
\DeclarePairedDelimiter\floor{\lfloor}{\rfloor}

\title{Otimização Combinatória em  \\ testes}
\author{Rafael Grisotto e Souza - RA 192765\\ Vinicius \\ Paulo Henrique Carvalho de Morais - RA 192877}



\begin{document}

\maketitle


\begin{questao}
\end{questao}

Resumo de Search-Based Software Testing.
https://ieeexplore.ieee.org/document/5954405/

Descrever EvoSuite (meta heurística)
http://www.evosuite.org/publications/

Alternativas para melhorar os resultados: meta heurística a, b, c, ...


Trabalho 2 começa aqui:
\cite{arcuri2017restful}
Projetar e rodar experimento com todas ou um subconjunto e comparar.





\newpage

\section{Referencial EvoSuite}
EvoSuite é uma ferramenta de geração de suítes de testes com alta cobertura e produz asserções \cite{fraser2011evosuite}. O EvoSuite usa uma abordagem baseada em pesquisa, integrando técnicas atuais tais como hybrid search, dynamic symbolic execution e testability transformation.

\textbf{Geração de suítes de testes:} EvoSuite usa abordagem de busca evolucionária que evolui os conjuntos de casos de testes de acorodo com a um critério de cobertura. A otimização prioriza o critério de cobertura ao invés objetivos locais que não resultam em diversidade influenciada pela ordem, dificuldade ou inviabilidade de objetivos individuais.

\textbf{Mutações baseadas em gerações de acerssões:} EvoSuite utiliza teste de mutação para produzir um conjunto reduzido de que maximiza a inclusão de defeitos em uma classe que são revelados pelos casos de testes. Isso ajuda dando feedback aos desenvolvedores de quais defeitos devem ser tratados, além de o atual comportamento para evitar regressões.



\section{Meta-Heurísticas}
Meta-heurística é um subcampo primário da otimização estocática, a qual consiste de algoritmos e técnicas que empregam algum grau de aleatoriedade para encontrar a solução ótima (ou a melhor possível) para problemas reconhecidamente difíceis \cite{luke2009essentials}. \citeauthor{luke2009essentials} ressalva que meta-heurística é um termo que pode gerar desentendimentos uma vez que não se trata de heuristica sobre (ou para) heurísticas, o que não é necessariamente verdade para todos os algoritmos. 

\end{document}


